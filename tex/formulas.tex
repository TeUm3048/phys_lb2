\section{Основные используемые формулы}
%=== 1 задание обработки
%=== 2 задание обработки
Период колебаний маятника:
\begin{equation}\label{mainEq:period}
    T=t/n .
\end{equation}

%=== 3 задание обработки
Момент инерции маятника:
\begin{equation}\label{mainEq:momentOfInertion}
    I=m g x_{\mathrm{c}} T_0^2 / 4 \pi^2 .
\end{equation}

%=== 4 задание обработки
Полная механическая энергия маятника:
\begin{equation}\label{mainEq:potentionEnergy}
    W = W_{\mathrm{pm}} \approx \frac{1}{2} m g x_{\mathrm{c}} \varphi_{\mathrm{m}}^2 .
\end{equation}

%=== 5 задание обработки
Приведенная длина маятника:
\begin{equation}\label{eq:approxLength}
    l_0=I / m x_c=g T_0^2 / 4 \pi^2 .
\end{equation}

%=== 6 задание обработки 
%=== 7 задание обработки
Положение центра масс:

%=== 8 задание обработки
Моменты инерции каждого из тел составного маятника и его полный момент инерции $I = \sum I_i$

\newpage