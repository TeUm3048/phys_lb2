\section{Вопросы}

9.	Напишите уравнение для кинетической и потенциальной энергии физического маятника. Найдите полную энергию. Какой характер сил, действующих на качающееся тело, консервативный или диссипативный?

Ответ:
Потенциальная энергия  равна:
\begin{equation}\label{equestionsEq:potentionEnergy}
    W_{\mathrm{p}}
    =m g h_{\mathrm{c}}
    =m g x_{\mathrm{c}}\left(1-\cos \varphi\right)
    =2 m g x_{\mathrm{c}} \sin ^2 \frac{\varphi}{2}
    \approx \frac{1}{2} m g x_{\mathrm{c}} \varphi^2
\end{equation}
где $h_c$ - высота поднятия центра масс маятника при его максимальном отклонении от положения равновесия, $x_{\mathrm{c}}$ - положение центра масс маятника относительно его точки подвеса, $\varphi$ - угол отклонения маятника от положения равновесия.

Кинетическая энергия маятника
\begin{equation}\label{equestionsEq:keneticEnergy}
    W_{\mathrm{k}}=\frac{I \omega^2}{2}
\end{equation}

На качающееся тело действуют сила тяжести и сила упругости, которые является консервативными силами.


1.	Какие силы называются консервативными?

Ответ:

Консервативные силы — силы, работа которых по замкнутой траектории равна 0.

\newpage